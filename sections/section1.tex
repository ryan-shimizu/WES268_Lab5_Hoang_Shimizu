
\documentclass[../main.tex]{subfiles}
\begin{document}
\setlength\parindent{24pt}
\section[Coding Gain for Linear Block Codes]
{Part 1: Coding Gain for Linear Block Codes}
\hspace*{1.5em}\textbf{3. - }
{
For a range of $E_b/N_0$ values from 0dB to 10dB (i.e. eleven 1dB steps), measure the Bit
Error-Rate for the following "codes":
\par
}
\textbf{a. Uncoded}
\begin{figure}[H]
   \centering
   \includegraphics[width=0.7\textwidth]{../lab_ss/WES268B_Lab5_Part1-3a_uncoded.png}
   \caption{1.3.a - Uncoded Bit Error Rates under AWGN}
\end{figure}
\par

\textbf{b. n-Repetition(3,1,3)}
\begin{figure}[H]
   \centering
   \includegraphics[width=0.7\textwidth]{../lab_ss/WES268B_Lab5_Part1-3b_n-repeat.png}
   \caption{1.3.b - n-Repetition(3,1,3) Bit Error Rates under AWGN}
\end{figure}
\par

\textbf{c. Hamming (7,4,3)}
\begin{figure}[H]
   \centering
   \includegraphics[width=0.7\textwidth]{../lab_ss/WES268B_Lab5_Part1-3c_hamming.png}
   \caption{1.3.c - Hamming (7,4,3) Bit Error Rates under AWGN}
\end{figure}
\par

\textbf{d. Golay (24,12,8)}
\begin{figure}[H]
   \centering
   \includegraphics[width=0.7\textwidth]{../lab_ss/WES268B_Lab5_Part1-3d_golay.png}
   \caption{1.3.d - Golay (24,12,8) Bit Error Rates under AWGN}
\end{figure}
\par

\end{document}