
\documentclass[../main.tex]{subfiles}
\begin{document}
\section[Channel Estimation using a PN Sequence]
{Part 1: Channel Estimation using a PN Sequence}
\subsection{Impulse Response Estimation of a Multipath Channel}
\subsubsection{Basic Measurements}
\hspace*{1.5em}\textbf{5. - }
{Since we are convolving with a time-delayed version of the PN sequence, this introduces frequency-dependent
modulation where the main and delayed copies of the signal clash. This shifts the frequency at the DC Component
away from 0Hz. This shift is characterized by the delay $\tau$:
$$f_{DC}=\frac{1}{\tau} \approx \text{100Hz}$$
}
\newline
\hspace*{1.5em}\textbf{6. - }
\begin{figure}[H]
   \centering
   \includegraphics[width=0.7\textwidth]{../lab_ss/p1_1_1_56.png}
   \caption{1.1.1.5/6 - Frequency at DC Component and PN sequence spectrum}
\end{figure}

\noindent\hspace*{1.5em}\textbf{7/8. - } 
\begin{figure}[H]
   \centering
   \includegraphics[width=0.7\textwidth]{../lab_ss/p1_1_1_78.png}
   \caption{1.1.1.7/8 - Period of Delta spikes of autocorrelation graph}
\end{figure}
\hspace*{1.5em}\textbf{9/10. - } 
\begin{figure}[H]
   \centering
   \includegraphics[width=0.7\textwidth]{../lab_ss/p1_1_1_10_freqlock.png}
   \caption{1.1.1.9/10 - Period of Delta spikes of autocorrelation graph}
\end{figure}
{The frequency of the DC component we found in part 5 is related to the multipath delay we introduced. The other components,
such as the spectral line spacing and autocorrelation delta spike spacing are related to the length of the PN sequence itself.
The autocorrelation envelope is dependent on the pulse width of the PN sequence

When we enable the PLL, we saw that our frequency offset was corrected and the DC component shifted to 0Hz as expected. We also
saw that our autocorrelation spikes became more apparent because they are now aligned with the PN sequence timing. The PLL corrects 
for the phase drift that misaligned our PN sequence timing, as well as the frequency offset. This lets us see the delta spikes more clearly.}

\subsubsection{Fading Measurements and Equalization}
\hspace*{1.5em}\textbf{2. - }
\begin{figure}[H]
   \centering
   \includegraphics[width=0.7\textwidth]{../lab_ss/p1_1_2_2.png}
   \caption{1.1.2.2 - Gains of Direct and Echo Path Terms}
\end{figure}

\hspace*{1.5em}\textbf{3. - } 
\begin{figure}[H]
   \centering
   \includegraphics[width=0.7\textwidth]{../lab_ss/p1_1_2_34_GOATED.png}
   \caption{1.1.2.3 - Gains of Direct and Echo Path Terms using fading parameters at receiver}
\end{figure}
{When we input the measured fading parameters in the receiver, we see that the gains of the direct and echo path terms
align more closely with the measured values from part 2.}
\newline
\noindent\hspace*{1.5em}\textbf{4. - } 
{Surprisngly, our setup did not need further adjustment as the gains of the direct and echo path terms already aligned well with the measured values from part 2. In 
a real multipath system, we would likely need to iteratively adjust the equalizer parameters to minimize error between the estimated and measured path gains.}
\newline
\hspace*{1.5em}\textbf{5a. - } 
\begin{figure}[H]
   \centering
   \includegraphics[width=0.7\textwidth]{../lab_ss/p1_1_2_5a2.png}
   \caption{1.1.2.5a - Gains of Direct and Echo Path Terms, PN Order of 6}
\end{figure}
\begin{figure}[H]
   \centering
   \includegraphics[width=0.7\textwidth]{../lab_ss/p1_1_2_5a3.png}
   \caption{1.1.2.5a - Gains of Path Terms using transmitter fading parameters, PN Order of 6}
\end{figure}
\begin{figure}[H]
   \centering
   \includegraphics[width=0.7\textwidth]{../lab_ss/p1_1_2_5a4_check.png}
   \caption{1.1.2.5a - Gains of Path Terms using measured fading parameters, PN Order of 6}
\end{figure}
\noindent\hspace*{1.5em}\textbf{5b. - } 
\begin{figure}[H]
   \centering
   \includegraphics[width=0.7\textwidth]{../lab_ss/p1_1_2_5b2.png}
   \caption{1.1.2.5b - Gains of Direct and Echo Path Terms, PN Order of 7}
\end{figure}
\begin{figure}[H]
   \centering
   \includegraphics[width=0.7\textwidth]{../lab_ss/p1_1_2_5b34.png}
   \caption{1.1.2.5b - Gains of Path Terms using transmitter fading parameters, PN Order of 7}
\end{figure}
\noindent\hspace*{1.5em}\textbf{6. - } 
{The spikes in our power spectrum graph from part 5 correspond to the periodicity of the PN sequence we transmitted. The period is related to the length
of the PN sequence used, which determines how close or far-apart the spectral lines are spaced.}
\newline
\hspace*{1.5em}\textbf{7. - } 
{As we increase the length of the PN sequence, we observe the spectral lines in the power spectrum grow closer together. Longer PN sequences also have practical uses as well.
Longer PN sequences have sharper autocorrelation spikes with lower sidelobes, which improves estimation accuracy. Having longer PN sequences also allows multiple users to share the same
band. This comes at the cost of computational complexity and processing time, so this is usually reserved for situations that have low SNR or require multiple access.}
\newline
\hspace*{1.5em}\textbf{8. - } 
{When we increase the gain, A, the autocorrelation spikes become taller and our power spectrum shows deeper nulls. This is a result of the stronger echo path interfering with the direct path.
When we increase the delay of our echo path, T, the echo peaks shift outward in the autocorrelation graph. In our power spectrum, we see the nulls become closer together.}

% Write-Up section
\subsection{Write-Up}
\subsubsection{}
\textit{Describe how you would create an FIR approximation of the equalization filter $g(n)$ denoted
g(n). Assuming the following H(z):
$$H(z) = 1 + Az^{-T}$$
$$G(z) = \frac{1}{1 + Az^{-T}}$$
You may assume that $|A| < 1$, and $T$ is a positive integer. Hint: What can you say about
$g(n)$ for large values of $n > N$.
}

\noindent\textbf{Answer.}  
{
   The given channel and equalizer transfer functions are:
   $$H(z)=1+Az^{-T}$$
   $$G(z)=\frac{1}{1+Az^{-T}}$$
   $$G(z)=1-Az^{-T}+A^2z^{-2T}+...$$
   which is an infinite series. To create an FIR approximation of the equalization filter $g(n)$, 
   we can truncate this series after a finite number of terms, say $N$ terms. 
   This means we will only consider the first $N$ terms of the series expansion of $G(z)$:
   $$G(z) \approx 1 - Az^{-T} + A^{2}z^{-2T}$$
   For large values of N, we approach the true equalizer response. But for practicality, we must choose
   a finite N taps for our FIR filter.
}
\newline
\textit{Now, for values $A = 0.5$ and $T = 4$, find a $\hat{g}(n)$.}
\newline
\noindent\textbf{Answer.}  
Given our previous equalization filter derivation from earlier:
$$g(n)=\delta(n)-A\delta(n-T)+A^2\delta(n-2T)-A^3\delta(n-3T)+...$$
$$g(n)=\delta(n)-0.5\delta(n-4)+0.5^2\delta(n-8)-0.5^3\delta(n-12)+...$$
Again, we truncate as before:
$$\hat{g}(n)=\delta(n)-0.5\delta(n-4)+0.25\delta(n-8)-0.125\delta(n-12)$$
\subsubsection{}
\textit{
Based on your observations from this lab, describe the relationship between invertibility of
a channel $H(z)$ in respect to its poles and/or zero locations. In other words, what kind of
channels $h[n]$ are invertible (i.e. does a causal equalizer filter $g[n]$ exist?) and non-invertible?
}
\newline
\noindent\textbf{Answer.}  
The channel $H(z)$ is invertible if none of the zeros of $H(z)$ lie inside the unit circle. 
This is because the equalizer $G(z)$ is given by $$G(z) = 1/H(z)$$ and if $H(z)$ has zeros inside the unit circle, 
then $G(z)$ will have poles inside the unit circle, making it non-causal and unstable.

\subsubsection{}
\textit{
a) Suppose that we wish to estimate the channel using a particular training sequence called the
"WES" sequence. All you know about this code is that it has very good auto-correlation
properties $ \Sigma w(k)w^{*}(n + k) = \delta(n)$ and the following quasi-periodic property:
$$w(n) = \begin{cases}
w(n+mL) & 0 \leq m \leq 9,\\
0  & \text{otherwise}
\end{cases}$$
Effectively, the "WES" sequence is a length $L$ sequence that repeats itself 10 times. Now,
assuming the input to your receiver is the "WES" sequence $w(n)$ with a frequency offset of
$f_0$
$$x(n) = w(n)e^{j2\pi f_0 n}$$
Find the cross-correlation between the received sequence $x(n)$ and the "WES" sequence
$w(n)$, denoted $r_{xw}(n)$
$$r_{xw}(n) = \sum_{k} x(n + k) w^{*}(k)$$
for values of $n = \{0, L, . . . , 9L\}$, you may assume that the term A is a constant where
$$A = \sum_{k} |w(k)|^2e^{j2\pi f_0 k}$$
}
\newline
\noindent\textbf{Answer.}  
\medskip
Assume the receiver input is
\[
x(n)=w(n)e^{j2\pi f_0 n},
\]
and the WES code satisfies the quasi-periodic property over the 10 repeats,
\[
w(n+mL)=w(n), \quad m=0,\ldots,9.
\]
With a single length-$L$ reference in the correlator,
\[
r_{xw}(n)=\sum_{k=0}^{L-1} x(n+k)w^*(k)
=\sum_{k=0}^{L-1} w(n+k)e^{j2\pi f_0(n+k)}w^*(k).
\]
Factor out the $n$-dependent exponential:
\[
r_{xw}(n)=e^{j2\pi f_0 n}\sum_{k=0}^{L-1} w(n+k)w^*(k)e^{j2\pi f_0 k}.
\]
Evaluate at $n=mL$ and use $w(mL+k)=w(k)$:
\[
\begin{aligned}
r_{xw}(mL)
&=e^{j2\pi f_0 mL}\sum_{k=0}^{L-1} w(k)w^*(k)e^{j2\pi f_0 k}\\
&=e^{j2\pi f_0 mL}\sum_{k=0}^{L-1} |w(k)|^2 e^{j2\pi f_0 k}.
\end{aligned}
\]
Define the constant (over one repetition)
\[
A=\sum_{k=0}^{L-1} |w(k)|^2 e^{j2\pi f_0 k}.
\]
Thus,
\[
r_{xw}(mL)=A e^{j2\pi f_0 mL}, \quad m=0,\ldots,9.
\]
\newline\newline


\noindent\textit{
b) Using your calculation of $r(0)$, and the fact that $|w(k)|^2 = 1$, comment on the effect that
frequency offset has on the magnitude of the cross-correlation.
}
\newline
\noindent\textbf{Answer.}  
\medskip
From part 3A,
\[
r_{xw}(0)=A=\sum_{k=0}^{L-1} |w(k)|^2 e^{j2\pi f_0 k}.
\]
With $|w(k)|^2=1$ over one length-$L$ block,
\[
r_{xw}(0)=\sum_{k=0}^{L-1} e^{j2\pi f_0 k}
=e^{j\pi f_0(L-1)}\frac{\sin(\pi f_0 L)}{\sin(\pi f_0)}.
\]
Hence
\[
|r_{xw}(0)|=\left|\frac{\sin(\pi f_0 L)}{\sin(\pi f_0)}\right|.
\]
The magnitude is maximized at $f_0=0$ (equal to $L$ for this normalization) and decreases as $|f_0|$ increases due to phasor cancellation. The correlation peak magnitude has nulls at
\[
f_0=\pm \frac{1}{L}, \pm \frac{2}{L}, \ldots,
\]
which is consistent with the sinusoidal-like envelope behavior observed when a nonzero frequency offset is present.
\newline

\noindent\textit{
c) Explain how you would calculate the frequency offset from your calculated values of
$r_{xw}(mL)$.
}
\newline
\noindent\textbf{Answer.}  
\medskip
From part 3A,
\[
r_{xw}(mL)=A e^{j2\pi f_0 mL}.
\]
Thus the phase of the correlation samples is linear in $m$. Using the ratio of adjacent peaks,
\[
\frac{r_{xw}((m+1)L)}{r_{xw}(mL)}=e^{j2\pi f_0 L},
\]
an estimator is
\[
\widehat{f}_0=\frac{1}{2\pi L}\angle\!\big(r_{xw}((m+1)L)\,r_{xw}^*(mL)\big),
\]
and in practice this can be averaged over $m=0,\ldots,8$ to reduce noise sensitivity.
\newline\newline


\noindent\textit{
d) What is the maximum frequency offset $f_{max}$ that can be tracked using the cross-correlation approach and the "WES" code $w(n)$ of length $L$ (Hint: Think of the nyquist
sampling theorem).
}
\newline
\noindent\textbf{Answer.}  
\medskip
The sequence of correlation peaks
\[
r_{xw}(mL)=A e^{j2\pi f_0 mL}
\]
samples a complex exponential once per length-$L$ block. In the index $m$, the normalized frequency is $f_0 L$ cycles per sample of $m$. To avoid aliasing,
\[
|f_0 L|<\frac{1}{2}
\quad\Rightarrow\quad
|f_0|<\frac{1}{2L}.
\]
Thus the maximum unambiguous offset is
\[
f_{\max}=\frac{1}{2L}\ \text{cycles/sample}.
\]
or
\[
f_{\max,\mathrm{Hz}}=\frac{f_s}{2L}.
\]
\newline\newline

\end{document}