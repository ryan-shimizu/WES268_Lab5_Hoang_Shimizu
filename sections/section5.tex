
\documentclass[../main.tex]{subfiles}
\usepackage{array}
\begin{document}
\setlength\parindent{0pt}
\section[Additional Questions]
{Part 5: Additional Questions}
\par
\textbf{1. -
Consider the 8-PSK constellation given by symbol values below: }
\begin{figure}[H]
   \centering
   \includegraphics[width=0.7\textwidth]{../lab_ss/Figure3_1.png}
\end{figure}
\par
\textbf{
Can you construct a gray-coded symbol mapper table for this constellation? Justify your
answer (i.e. if it's possible, create the symbol mapping table, otherwise explain why it's
not possible).
}
\par

{
   Yes, it is possible to create a Gray-coded symbol mapper table for this
   constellation because it is uniformly spaced with a phase interval of $\pi/4$ and has exactly 3 nearest neighbors (one for each possible bit flip).
   We can use a 3-bit mapping since we have $2^3$ symbols:
   $$000 \rightarrow 001 \rightarrow 011 \rightarrow 010 \rightarrow 110 \rightarrow 111 \rightarrow 101 \rightarrow 100$$
}
\textbf{2. - 
Consider the 8-QAM constellation given by symbol values below:
\begin{figure}[H]
   \centering
   \includegraphics[width=0.7\textwidth]{../lab_ss/Figure3_2.png}
\end{figure}
Can you construct a gray-coded symbol mapper table for this constellation? Justify your
answer (i.e. if it's possible, create the symbol mapping table, otherwise explain why it's
not possible)
}
\par

{
   No, it is not possible to create a Gray-coded symbol mapper table for this constellation.
   In the above configuration, the symbols in the center of the constellation have 
   4 nearest neighbors, which is impossible to map with only 3 bits to work with.
}


\end{document}