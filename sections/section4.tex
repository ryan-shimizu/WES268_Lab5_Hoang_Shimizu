
\documentclass[../main.tex]{subfiles}
\begin{document}
\setlength\parindent{0pt}
\section[Post Lab]
{Part 4: Post Lab}
\par
\indent\textbf{1. - }
{
On the same graph, plot the Uncoded Bit Error Rate and Coded Bit Error Rate for the
n-Repetition (3,1) and Hamming (7,4), and Golay (24,12) codes.
\par
}
\textbf{n-Repetition (3,1)}
% \begin{figure}[H]
%    \centering
%    \includegraphics[width=0.7\textwidth]{../lab_ss/WES268B_Lab5_Part3-2a_gray-coded.png}
%    \caption{3.2.a - "Gray Coded" QPSK Bit Error Rates under AWGN}
% \end{figure}
\par

\textbf{Hamming (7,4)}
% \begin{figure}[H]
%    \centering
%    \includegraphics[width=0.7\textwidth]{../lab_ss/WES268B_Lab5_Part3-2a_gray-coded.png}
%    \caption{3.2.a - "Gray Coded" QPSK Bit Error Rates under AWGN}
% \end{figure}
\par

\textbf{Golay (24,12)}
% \begin{figure}[H]
%    \centering
%    \includegraphics[width=0.7\textwidth]{../lab_ss/WES268B_Lab5_Part3-2a_gray-coded.png}
%    \caption{3.2.a - "Gray Coded" QPSK Bit Error Rates under AWGN}
% \end{figure}
\par

\textbf{Give an explanation for the cross-over point of the uncoded and coded ber curves.}
\par
\textbf{2. - }
{
On the same graph, plot the Uncoded Bit Error Rate and Coded Bit Error Rate for the
LDPC code for range of values you used for \textit{Max Iterations}. Are there any changes in coding
gain for different values of \textit{Max Iterations}?
}
\par

\textbf{3. - }
{
On the same graph, plot the Uncoded Bit Error Rate for QPSK with both "gray" and "non
gray" codes. Explain the difference between the two curves.
}
\par

\textbf{4. - }
{
What is the measured coding gain for the error correction codes from parts 1? Which error
correction code is better? Why?
}
\par

\textbf{5. - }
{
Compare the measured coding gains from part 1 to the expression for asymptotic coding gain
$G$, where
$$G \approx R_c(t+1)$$
$$t=\text{floor}(\frac{d_{min}-1}{2})$$
Do they match? Why or why not?
}
\par

\textbf{6. - }
{
What are the consequences of
using an \textunderscore{even} value of $n$ for an n-repetition (n,1) code?
}
\par

\end{document}